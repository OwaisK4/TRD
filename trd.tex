\documentclass[letterpaper,12pt]{article} %{amsart}
\usepackage[pdftex]{geometry}
\usepackage[utf8]{inputenc}
\usepackage[english]{babel}

% Use Times New Roman
\usepackage{mathptmx}

\usepackage{listings}
\usepackage{amssymb}
\usepackage{amsmath}
\usepackage{multicol}
\usepackage{courier}
\usepackage{fancyhdr}

\lstset{columns=fullflexible}
\lstset{breaklines=true, breakatwhitespace=true}
\lstset{basicstyle=\scriptsize\ttfamily}
\lstset{showspaces=false}
\lstset{showstringspaces=false}

\geometry{landscape}
\setlength{\topmargin}{-0.25in}
\setlength{\oddsidemargin}{0in}
\setlength{\evensidemargin}{0in}
\setlength{\columnsep}{0.5in}
\setlength{\columnseprule}{0.2pt}
\textwidth 9.5in

\pagestyle{fancy}
\lhead{FAST NUCES, Karachi}
\chead{}
\rhead{\thepage}
\lfoot{}
\cfoot{}
\rfoot{}

\newcommand{\includecpp}[2]{
  \subsection{#1}\
  \lstinputlisting[language=c++]{#2}
}

\begin{document}
  \thispagestyle{fancy}
  \pagenumbering{Roman}
  
  \begin{center}

    \Huge\textsc{ACM-ICPC Team Reference Document}

    \huge The Codists
    
    \vspace{0.50cm}
  \end{center}
  
  \begin{multicols*}{2}
    \tableofcontents
  \end{multicols*}
  
  \enlargethispage*{\baselineskip}
  \pagebreak
  
  % == CONTENTS == %
  
  \pagenumbering{arabic}
  \setcounter{page}{1}
  \begin{multicols*}{2}
    \input{contents.tex}
    \text{Number of divisors:}
    \[
    d(n) = (e_1 + 1) \cdot (e_2 + 1) \cdots (e_k + 1)
    \]
    \text{Sum of divisors:}
    \[
    \sigma(n) = \frac{p_1^{e_1+1} - 1}{p_1 - 1} \cdot \frac{p_2^{e_2+1} - 1}{p_2 - 1} \cdots \frac{p_k^{e_k+1} - 1}{p_k - 1}
    \]
    \text{Substring hashing:}
    When polynomial hashing is used, we can calculate the hash value of any substring of a string \( s \) in \( O(1) \) time after an \( O(n) \) time preprocessing. The idea is to construct an array \( h \) such that \( h[k] \) contains the hash value of the prefix \( s[0 \dots k] \). The array values can be recursively calculated as follows:
    \[
    h[0] = s[0]
    \]
    \[
    h[k] = (h[k - 1] A + s[k]) \mod B
    \]
    In addition, we construct an array \( p \) where \( p[k] = A^k \mod B \):
    \[
    p[0] = 1
    \]
    \[
    p[k] = (p[k - 1] A) \mod B.
    \]
    Constructing the above arrays takes \( O(n) \) time. After this, the hash value of any substring \( s[a \dots b] \) can be calculated in \( O(1) \) time using the formula:
    \[
    (h[b] - h[a - 1] p[b - a + 1]) \mod B
    \]
    assuming that \( a > 0 \). If \( a = 0 \), the hash value is simply \( h[b] \).
  \end{multicols*}
\end{document}
